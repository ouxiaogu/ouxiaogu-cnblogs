% pandoc -s eq1.tex -o eq1.md
% pandoc -s blog1.md -o blog1.html
\documentclass[11pt,a4paper]{article}
\usepackage{amsmath}
\usepackage{fontspec}
\usepackage{indentfirst} % 首行缩进 indentfirst.sty
\setmainfont{SimSun}
\setlength {\parindent }{2em} % 设置首行缩进量
\addtolength {\parskip }{3pt} % 设置段间距
\linespread {1.3} %一倍半行距

%\begin{CJK}{GBK}{song}% 设置字体
\begin{document}

% sum的上标用大括号{}表示
    \begin{equation}
      B2U_w\left( \overrightarrow{x} \right) \doteq \sum_{i=0}^{w-1} x_i 2^i \label{eq1}
    \end{equation}
    \begin{equation}
      B2T_w\left( \overrightarrow{x} \right) \doteq -x_{w-1} 2^{w-1} + \sum_{i=0}^{w-2} x_i 2^i \label{eq2}
    \end{equation}
    对于位模式 $\overrightarrow{x}$ ,对比 (\ref{eq1}) (\ref{eq2}) , 计算两者之差,我们就可以得到
   $$ B2U_w\left( \overrightarrow{x} \right) - B2T_w\left( \overrightarrow{x} \right) = x_{w-1}  \left( 2^{w-1} - \left( -2^{w-1} \right) \right)  =  x_{w-1} 2^w $$ ,这样就得到了 $$B2U_w\left( \overrightarrow{x} \right) = x_{w-1} 2^w + B2T_w\left( \overrightarrow{x} \right)$$ 。若令 $$\overrightarrow{x} = T2B_w \left( x \right) $$ ,则其反函数为 $$ x = B2T_w \left( \overrightarrow{x} \right) $$ 。 由前面三式以及$T \rightarrow B \rightarrow U$变换的传递性,可得:

    \begin{equation}
      B2U_w\left( T2B_w \left( x \right) \right) = T2U_w \left( x \right) = x_{w-1} 2^w + x
    \end{equation}

    这个关系对于理解“有符数变换得到的无符数也即补码”很有用。
    \begin{equation}
       T2U_w \left( x \right)=\left\{
        \begin{aligned}
          & x+2^w, & x<0,x_{w-1}=1  \\
          & x, & x \geq 0,x_{w-1}=0
        \end{aligned}
        \right.
    \end{equation}
    下图说明了 $T2U$ 的转换行为:对于非负数($x \geq 0$), $T2U$ 保留原值 ; 对于负数($x<0$),$T2U$ 被装换为一个大于$2^{w-1}$的正数。\\
% 无标号的版本
%    \[
%      T2U_w \left( x \right) =
%          \begin{cases}
%            x+2^w,\quad {x<0 , x_{w-1}=1 } \\
%            x,\quad {x \neq 0 , x_{w-1}=0 }
%          \end{cases}
%      \]
    反过来,我们再来推导无符号数 $u$ 和与之对应的有符号数 $U2T_w \left( u \right) $之间的关系。
上一节我们已经得到 $$B2T_w\left( \overrightarrow{u} \right) = B2U_w\left( \overrightarrow{u} \right) - u_{w-1} 2^w $$ 。 若令 $$\overrightarrow{u} = U2B_w \left( u \right) $$ ,则其反函数为 $$ u = B2U_w \left( \overrightarrow{x} \right) $$ 。由前面三式以及 $T \rightarrow B \rightarrow U$ 变换的传递性,可得:

    \begin{equation}
      B2T_w\left( U2B_w \left( u \right) \right) = U2T_w \left( u \right) = u - u_{w-1} 2^w
    \end{equation}

    在 $u$ 原始的无符号表示法中,最高位$u_{w-1}$决定了 $u$ 是否大于或等于 $2^{w-1}$, 无符数 $u$ 到有符数的装换分段表示为:
    \begin{equation}
       U2T_w \left( u \right)=\left\{
        \begin{aligned}
          & u, & u<2^{w-1},x_{w-1}=0  \\
          & u-2^w,  & x \leq 0,x_{w-1}=1
        \end{aligned}
        \right.
    \end{equation}
    下图说明了 $U2T$ 转换行为:对于小的数($u<2^{w-1}$), $U2T$ 保留原值 ; 对于大的数($u \geq 2^{w-1}$),$U2T$ 被装换为一个负数。
%\end{CJK}
\end{document}

